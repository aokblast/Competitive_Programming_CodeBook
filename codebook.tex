% !TeX program = xelatex 

\documentclass{article}
\usepackage{xeCJK}
\usepackage{listings}
\usepackage{xcolor}
\usepackage{geometry}
\geometry{a4paper,scale=0.9}

\definecolor{codegreen}{rgb}{0,0.6,0}
\definecolor{codegray}{rgb}{0.5,0.5,0.5}
\definecolor{codepurple}{rgb}{0.58,0,0.82}
\definecolor{backcolour}{rgb}{0.95,0.95,0.92}

\lstdefinestyle{mystyle}{
    backgroundcolor=\color{backcolour},   
    commentstyle=\color{codegreen},
    keywordstyle=\color{magenta},
    numberstyle=\tiny\color{codegray},
    stringstyle=\color{codepurple},
    basicstyle=\ttfamily\footnotesize,
    breakatwhitespace=false,         
    breaklines=true,                 
    captionpos=b,                    
    keepspaces=true,                 
    numbers=left,                    
    numbersep=5pt,                  
    showspaces=false,                
    showstringspaces=false,
    showtabs=false,                  
    tabsize=2
}

\lstset{style=mystyle}



\title{代码簿}
\begin{document}
\maketitle

\begin{enumerate}

{\bf \LARGE \item  杂项}

	\begin{itemize}

	{\bf \item  Merge Sort}
	\lstinputlisting[language=C++]{./other/MergeSort.cpp}

	{\bf \item  Disjoint Set}
	\lstinputlisting[language=C++]{./other/DisjointSet.cpp}

	\end{itemize}

{\bf \LARGE \item  图论}


	\begin{itemize}
	
	{\bf \item  Eular Path}	
	\lstinputlisting[language=C++]{./graph/EularPath.cpp}
	
	{\bf \item  拓扑排序}	
	\lstinputlisting[language=C++]{./graph/TopoSort.cpp}	
	
	{\bf \item  多元最短路}
	\lstinputlisting[language=C++]{./graph/Floyd-Warshal.cpp}	
	
	{\bf \item  单元最短路}
	\lstinputlisting[language=C++]{./graph/Dijkstra.cpp}
	
	{\bf \item  找负环}
	\lstinputlisting[language=C++]{./graph/Bellman-Ford.cpp}
	
	\end{itemize}

{\bf \LARGE \item  动态规划}

	\begin{itemize}
	
	{\bf \item 区间覆盖}`	
	\lstinputlisting[language=C++]{./dp/IntervalCoverage.cpp}	
	
	{\bf \item  LIS}
	\lstinputlisting[language=C++]{./dp/LIS.cpp}	
	
		
	
	\end{itemize}


\end{enumerate}

\end{document}
